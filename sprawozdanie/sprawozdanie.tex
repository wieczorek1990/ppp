\documentclass[a4paper]{article}
\usepackage[polish]{babel}
\usepackage{polski}
\usepackage[T1]{fontenc}
\usepackage[utf8]{inputenc}
\usepackage{amsmath}
\usepackage{graphicx,color}
\usepackage{float}
\usepackage{cite}
\usepackage{url}
\usepackage[pdftex,hyperfootnotes=false,pdfborder={0 0 0}]{hyperref}
\usepackage{indentfirst}
\usepackage{subfig}
\usepackage{rotating}
\frenchspacing
%zmiana rozmiarów ramki tekstowej
\addtolength{\hoffset}{-2cm}
\addtolength{\textheight}{4cm}
\addtolength{\textwidth}{4cm}
\addtolength{\voffset}{-2cm}
%Sprawdzanie pisowni: aspell -t -l pl -c sprawozdanie.tex

\begin{document}
\thispagestyle{empty}
\begin{center}
{\LARGE{Sprawozdanie\\}}
{\large{Zadanie 4\\}}
\vspace{3ex}
\today\\
\vspace{3ex}
\begin{tabular}{llr}
\textbf{Łukasz Wieczorek} & inf94385 & wieczorek1990@gmail.com\\
\textbf{Maciej Graszek} & inf96292 & maciej.graszek@wp.pl\\
\end{tabular}
\end{center}
\newpage
\section{Model matematyczny}
\[max: przychod - koszty\]
\[przychod = sprzedaz \cdot 1,2\]
\[sprzedaz = \sum_{i} sprzedaz_i \cdot 300\]
\[sprzedaz_i \leq \sum_{j} transport_{ji}\]
\[sprzedaz_i \leq popyt_i\]
\[transport_{ji} \geq 1000 \cdot x_{ji}\]
\[transport_{ji} \leq zdolnosc\_produkcji_j \cdot x_{ji}\]
\[transport_{ji} \leq wielkosc\_produkcji_{ji}\]
\[\sum_{i} wielkosc\_produkcji_{ji} \geq 10\% \cdot zdolnosc\_produkcji_j \cdot x_j\]
\[\sum_{i} wielkosc\_produkcji_{ji} \leq \cdot zdolnosc\_produkcji_j \cdot x_j\]
\[koszty = (koszty\_transportu + koszty\_produkcji) \cdot 300\]
\[koszty\_produkcji = \sum_i \sum_j wielkosc\_produkcji_{ji} \cdot koszt\_produkcji_j\]
\[koszty\_transportu = \sum_i \sum_j transport_{ji} \cdot koszt\_transportu_{ji}\]
\section{Opis ograniczeń}
\subsection{\(sprzedaz_i \leq \sum_{j} transport_{ji}\)}
Sprzedaż na danym rynku jest mniejsza lub równa transportom, które dotarły z wszystkich piekarni.
\subsection{\(sprzedaz_i \leq popyt_i\)}
Sprzedaż na danym rynku jest mniejsza lub równa popytowi na tym rynku.
\subsection{\(transport_{ji} \geq 1000 \cdot x_{ji}\)}
Wielkość każdego transportu musi być większa od 1000kg, jeżeli on istnieje.
\subsection{\(transport_{ji} \leq zdolnosc\_produkcji_j \cdot x_{ji}\)}
Transport jest mniejszy lub równy zdolności produkcji, jeśli została uruchomiona.
\subsection{\(transport_{ji} \leq wielkosc\_produkcji_{ji}\)}
Transport jest mniejszy lub równy wielkości produkcji.
\subsection{\(\sum_{i} wielkosc\_produkcji_{ji} \geq 10\% \cdot zdolnosc\_produkcji_j \cdot x_j\)}
Produkcja w danej piekarni musi być większa od \(10\%\) zdolności produkcji danej piekarni, jeżeli została uruchomiona.
\subsection{\(\sum_{i} wielkosc\_produkcji_{ji} \leq \cdot zdolnosc\_produkcji_j \cdot x_j\)}
Wielkość produkcji danej piekarni jest mniejsza lub równa od zdolności produkcji, jeśli została uruchomiona.
\section{Wyniki}
\begin{table}[H]
\centering
\begin{tabular}{cccccc}
Typ\textbackslash Rok & 1 & 2 & 3 & 4 & 5\\
Bez Poznania & 18771000 & 20451600 & 22091220 & 23636346 & 24452817\\
Z Poznaniem & 21417000 & 23491200 & 25518000 & 27487200 & 29272350\\
\end{tabular}
\caption{NCF}
\label{tab:ncf}
\end{table}
W poniższej tabeli przedstawiono warianty budowy w różnych latach.
Legenda:
\begin{itemize}
\item bez -- w danym roku zakład w Poznaniu nie istnieje,
\item z -- w danym roku zakład w Poznaniu istnieje,
\item z-P -- w danym roku zakład w Poznaniu został wybudowany.
\end{itemize}
\begin{table}[H]
\centering
\begin{tabular}{cccccc}
1 & 2 & 3 & 4 & 5 & 6\\
bez & z-P & bez & bez & bez & bez\\
bez & z   & z-P & bez & bez & bez\\
bez & z   & z   & z-P & bez & bez\\
bez & z   & z   & z   & z-P & bez\\
bez & z   & z   & z   & z   & z-P\\
\end{tabular}
\caption{Warianty}
\label{tab:warianty}
\end{table}
\begin{table}[H]
\centering
\begin{tabular}{ccccccc}
Wariant	& 1 & 2 & 3 & 4 & 5 & 6\\
NPV & 82781394 & 86749245 &	85303245 & 83703593 & 82019828 & 80276848\\
Okres zwrotu & nd. &	4 & 3 & 3 & 3 & $2^*$\\
\end{tabular}
\caption{Wyniki dla wariatnów}
\label{tab:wyniki}
\end{table}
* (dla 6-ciu lat)
\section{Wnioski}
Zarekomendowalibyśmy budowę piekarni w Poznaniu w wariantach 2, 3, 4 (rok pierwszy, drugi, trzeci) ze względu na większą wartość NPV niż dla wariantu 1. Inwestycję należy zrealizować w wariancie 2 tj. w pierwszym roku by uzyskać największą wartość NPV. Inwestycję można zrealizować w roku drugim, trzecim lub czwartym by uzyskać najelpszą wartość okresu zwrotu (warianty 3, 4, 5) w analizie 5 letniej. Dla analizy 6-cio letniej najlepszą wartość okresu zwrotu osiąga wariant 6 (budowa w piątym roku). 
\bibliography{bibliografia}{}
\bibliographystyle{plain}
\end{document} 